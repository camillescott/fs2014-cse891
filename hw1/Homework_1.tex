
% Default to the notebook output style

    


% Inherit from the specified cell style.




    
\documentclass{article}

    
    
    \usepackage{graphicx} % Used to insert images
    \usepackage{adjustbox} % Used to constrain images to a maximum size 
    \usepackage{color} % Allow colors to be defined
    \usepackage{enumerate} % Needed for markdown enumerations to work
    \usepackage{geometry} % Used to adjust the document margins
    \usepackage{amsmath} % Equations
    \usepackage{amssymb} % Equations
    \usepackage[mathletters]{ucs} % Extended unicode (utf-8) support
    \usepackage[utf8x]{inputenc} % Allow utf-8 characters in the tex document
    \usepackage{fancyvrb} % verbatim replacement that allows latex
    \usepackage{grffile} % extends the file name processing of package graphics 
                         % to support a larger range 
    % The hyperref package gives us a pdf with properly built
    % internal navigation ('pdf bookmarks' for the table of contents,
    % internal cross-reference links, web links for URLs, etc.)
    \usepackage{hyperref}
    \usepackage{longtable} % longtable support required by pandoc >1.10
    \usepackage{booktabs}  % table support for pandoc > 1.12.2
    

    
    
    \definecolor{orange}{cmyk}{0,0.4,0.8,0.2}
    \definecolor{darkorange}{rgb}{.71,0.21,0.01}
    \definecolor{darkgreen}{rgb}{.12,.54,.11}
    \definecolor{myteal}{rgb}{.26, .44, .56}
    \definecolor{gray}{gray}{0.45}
    \definecolor{lightgray}{gray}{.95}
    \definecolor{mediumgray}{gray}{.8}
    \definecolor{inputbackground}{rgb}{.95, .95, .85}
    \definecolor{outputbackground}{rgb}{.95, .95, .95}
    \definecolor{traceback}{rgb}{1, .95, .95}
    % ansi colors
    \definecolor{red}{rgb}{.6,0,0}
    \definecolor{green}{rgb}{0,.65,0}
    \definecolor{brown}{rgb}{0.6,0.6,0}
    \definecolor{blue}{rgb}{0,.145,.698}
    \definecolor{purple}{rgb}{.698,.145,.698}
    \definecolor{cyan}{rgb}{0,.698,.698}
    \definecolor{lightgray}{gray}{0.5}
    
    % bright ansi colors
    \definecolor{darkgray}{gray}{0.25}
    \definecolor{lightred}{rgb}{1.0,0.39,0.28}
    \definecolor{lightgreen}{rgb}{0.48,0.99,0.0}
    \definecolor{lightblue}{rgb}{0.53,0.81,0.92}
    \definecolor{lightpurple}{rgb}{0.87,0.63,0.87}
    \definecolor{lightcyan}{rgb}{0.5,1.0,0.83}
    
    % commands and environments needed by pandoc snippets
    % extracted from the output of `pandoc -s`
    \DefineVerbatimEnvironment{Highlighting}{Verbatim}{commandchars=\\\{\}}
    % Add ',fontsize=\small' for more characters per line
    \newenvironment{Shaded}{}{}
    \newcommand{\KeywordTok}[1]{\textcolor[rgb]{0.00,0.44,0.13}{\textbf{{#1}}}}
    \newcommand{\DataTypeTok}[1]{\textcolor[rgb]{0.56,0.13,0.00}{{#1}}}
    \newcommand{\DecValTok}[1]{\textcolor[rgb]{0.25,0.63,0.44}{{#1}}}
    \newcommand{\BaseNTok}[1]{\textcolor[rgb]{0.25,0.63,0.44}{{#1}}}
    \newcommand{\FloatTok}[1]{\textcolor[rgb]{0.25,0.63,0.44}{{#1}}}
    \newcommand{\CharTok}[1]{\textcolor[rgb]{0.25,0.44,0.63}{{#1}}}
    \newcommand{\StringTok}[1]{\textcolor[rgb]{0.25,0.44,0.63}{{#1}}}
    \newcommand{\CommentTok}[1]{\textcolor[rgb]{0.38,0.63,0.69}{\textit{{#1}}}}
    \newcommand{\OtherTok}[1]{\textcolor[rgb]{0.00,0.44,0.13}{{#1}}}
    \newcommand{\AlertTok}[1]{\textcolor[rgb]{1.00,0.00,0.00}{\textbf{{#1}}}}
    \newcommand{\FunctionTok}[1]{\textcolor[rgb]{0.02,0.16,0.49}{{#1}}}
    \newcommand{\RegionMarkerTok}[1]{{#1}}
    \newcommand{\ErrorTok}[1]{\textcolor[rgb]{1.00,0.00,0.00}{\textbf{{#1}}}}
    \newcommand{\NormalTok}[1]{{#1}}
    
    % Define a nice break command that doesn't care if a line doesn't already
    % exist.
    \def\br{\hspace*{\fill} \\* }
    % Math Jax compatability definitions
    \def\gt{>}
    \def\lt{<}
    % Document parameters
    \title{Homework\_1}
    
    
    

    % Pygments definitions
    
\makeatletter
\def\PY@reset{\let\PY@it=\relax \let\PY@bf=\relax%
    \let\PY@ul=\relax \let\PY@tc=\relax%
    \let\PY@bc=\relax \let\PY@ff=\relax}
\def\PY@tok#1{\csname PY@tok@#1\endcsname}
\def\PY@toks#1+{\ifx\relax#1\empty\else%
    \PY@tok{#1}\expandafter\PY@toks\fi}
\def\PY@do#1{\PY@bc{\PY@tc{\PY@ul{%
    \PY@it{\PY@bf{\PY@ff{#1}}}}}}}
\def\PY#1#2{\PY@reset\PY@toks#1+\relax+\PY@do{#2}}

\expandafter\def\csname PY@tok@gd\endcsname{\def\PY@tc##1{\textcolor[rgb]{0.63,0.00,0.00}{##1}}}
\expandafter\def\csname PY@tok@gu\endcsname{\let\PY@bf=\textbf\def\PY@tc##1{\textcolor[rgb]{0.50,0.00,0.50}{##1}}}
\expandafter\def\csname PY@tok@gt\endcsname{\def\PY@tc##1{\textcolor[rgb]{0.00,0.27,0.87}{##1}}}
\expandafter\def\csname PY@tok@gs\endcsname{\let\PY@bf=\textbf}
\expandafter\def\csname PY@tok@gr\endcsname{\def\PY@tc##1{\textcolor[rgb]{1.00,0.00,0.00}{##1}}}
\expandafter\def\csname PY@tok@cm\endcsname{\let\PY@it=\textit\def\PY@tc##1{\textcolor[rgb]{0.25,0.50,0.50}{##1}}}
\expandafter\def\csname PY@tok@vg\endcsname{\def\PY@tc##1{\textcolor[rgb]{0.10,0.09,0.49}{##1}}}
\expandafter\def\csname PY@tok@m\endcsname{\def\PY@tc##1{\textcolor[rgb]{0.40,0.40,0.40}{##1}}}
\expandafter\def\csname PY@tok@mh\endcsname{\def\PY@tc##1{\textcolor[rgb]{0.40,0.40,0.40}{##1}}}
\expandafter\def\csname PY@tok@go\endcsname{\def\PY@tc##1{\textcolor[rgb]{0.53,0.53,0.53}{##1}}}
\expandafter\def\csname PY@tok@ge\endcsname{\let\PY@it=\textit}
\expandafter\def\csname PY@tok@vc\endcsname{\def\PY@tc##1{\textcolor[rgb]{0.10,0.09,0.49}{##1}}}
\expandafter\def\csname PY@tok@il\endcsname{\def\PY@tc##1{\textcolor[rgb]{0.40,0.40,0.40}{##1}}}
\expandafter\def\csname PY@tok@cs\endcsname{\let\PY@it=\textit\def\PY@tc##1{\textcolor[rgb]{0.25,0.50,0.50}{##1}}}
\expandafter\def\csname PY@tok@cp\endcsname{\def\PY@tc##1{\textcolor[rgb]{0.74,0.48,0.00}{##1}}}
\expandafter\def\csname PY@tok@gi\endcsname{\def\PY@tc##1{\textcolor[rgb]{0.00,0.63,0.00}{##1}}}
\expandafter\def\csname PY@tok@gh\endcsname{\let\PY@bf=\textbf\def\PY@tc##1{\textcolor[rgb]{0.00,0.00,0.50}{##1}}}
\expandafter\def\csname PY@tok@ni\endcsname{\let\PY@bf=\textbf\def\PY@tc##1{\textcolor[rgb]{0.60,0.60,0.60}{##1}}}
\expandafter\def\csname PY@tok@nl\endcsname{\def\PY@tc##1{\textcolor[rgb]{0.63,0.63,0.00}{##1}}}
\expandafter\def\csname PY@tok@nn\endcsname{\let\PY@bf=\textbf\def\PY@tc##1{\textcolor[rgb]{0.00,0.00,1.00}{##1}}}
\expandafter\def\csname PY@tok@no\endcsname{\def\PY@tc##1{\textcolor[rgb]{0.53,0.00,0.00}{##1}}}
\expandafter\def\csname PY@tok@na\endcsname{\def\PY@tc##1{\textcolor[rgb]{0.49,0.56,0.16}{##1}}}
\expandafter\def\csname PY@tok@nb\endcsname{\def\PY@tc##1{\textcolor[rgb]{0.00,0.50,0.00}{##1}}}
\expandafter\def\csname PY@tok@nc\endcsname{\let\PY@bf=\textbf\def\PY@tc##1{\textcolor[rgb]{0.00,0.00,1.00}{##1}}}
\expandafter\def\csname PY@tok@nd\endcsname{\def\PY@tc##1{\textcolor[rgb]{0.67,0.13,1.00}{##1}}}
\expandafter\def\csname PY@tok@ne\endcsname{\let\PY@bf=\textbf\def\PY@tc##1{\textcolor[rgb]{0.82,0.25,0.23}{##1}}}
\expandafter\def\csname PY@tok@nf\endcsname{\def\PY@tc##1{\textcolor[rgb]{0.00,0.00,1.00}{##1}}}
\expandafter\def\csname PY@tok@si\endcsname{\let\PY@bf=\textbf\def\PY@tc##1{\textcolor[rgb]{0.73,0.40,0.53}{##1}}}
\expandafter\def\csname PY@tok@s2\endcsname{\def\PY@tc##1{\textcolor[rgb]{0.73,0.13,0.13}{##1}}}
\expandafter\def\csname PY@tok@vi\endcsname{\def\PY@tc##1{\textcolor[rgb]{0.10,0.09,0.49}{##1}}}
\expandafter\def\csname PY@tok@nt\endcsname{\let\PY@bf=\textbf\def\PY@tc##1{\textcolor[rgb]{0.00,0.50,0.00}{##1}}}
\expandafter\def\csname PY@tok@nv\endcsname{\def\PY@tc##1{\textcolor[rgb]{0.10,0.09,0.49}{##1}}}
\expandafter\def\csname PY@tok@s1\endcsname{\def\PY@tc##1{\textcolor[rgb]{0.73,0.13,0.13}{##1}}}
\expandafter\def\csname PY@tok@sh\endcsname{\def\PY@tc##1{\textcolor[rgb]{0.73,0.13,0.13}{##1}}}
\expandafter\def\csname PY@tok@sc\endcsname{\def\PY@tc##1{\textcolor[rgb]{0.73,0.13,0.13}{##1}}}
\expandafter\def\csname PY@tok@sx\endcsname{\def\PY@tc##1{\textcolor[rgb]{0.00,0.50,0.00}{##1}}}
\expandafter\def\csname PY@tok@bp\endcsname{\def\PY@tc##1{\textcolor[rgb]{0.00,0.50,0.00}{##1}}}
\expandafter\def\csname PY@tok@c1\endcsname{\let\PY@it=\textit\def\PY@tc##1{\textcolor[rgb]{0.25,0.50,0.50}{##1}}}
\expandafter\def\csname PY@tok@kc\endcsname{\let\PY@bf=\textbf\def\PY@tc##1{\textcolor[rgb]{0.00,0.50,0.00}{##1}}}
\expandafter\def\csname PY@tok@c\endcsname{\let\PY@it=\textit\def\PY@tc##1{\textcolor[rgb]{0.25,0.50,0.50}{##1}}}
\expandafter\def\csname PY@tok@mf\endcsname{\def\PY@tc##1{\textcolor[rgb]{0.40,0.40,0.40}{##1}}}
\expandafter\def\csname PY@tok@err\endcsname{\def\PY@bc##1{\setlength{\fboxsep}{0pt}\fcolorbox[rgb]{1.00,0.00,0.00}{1,1,1}{\strut ##1}}}
\expandafter\def\csname PY@tok@kd\endcsname{\let\PY@bf=\textbf\def\PY@tc##1{\textcolor[rgb]{0.00,0.50,0.00}{##1}}}
\expandafter\def\csname PY@tok@ss\endcsname{\def\PY@tc##1{\textcolor[rgb]{0.10,0.09,0.49}{##1}}}
\expandafter\def\csname PY@tok@sr\endcsname{\def\PY@tc##1{\textcolor[rgb]{0.73,0.40,0.53}{##1}}}
\expandafter\def\csname PY@tok@mo\endcsname{\def\PY@tc##1{\textcolor[rgb]{0.40,0.40,0.40}{##1}}}
\expandafter\def\csname PY@tok@kn\endcsname{\let\PY@bf=\textbf\def\PY@tc##1{\textcolor[rgb]{0.00,0.50,0.00}{##1}}}
\expandafter\def\csname PY@tok@mi\endcsname{\def\PY@tc##1{\textcolor[rgb]{0.40,0.40,0.40}{##1}}}
\expandafter\def\csname PY@tok@gp\endcsname{\let\PY@bf=\textbf\def\PY@tc##1{\textcolor[rgb]{0.00,0.00,0.50}{##1}}}
\expandafter\def\csname PY@tok@o\endcsname{\def\PY@tc##1{\textcolor[rgb]{0.40,0.40,0.40}{##1}}}
\expandafter\def\csname PY@tok@kr\endcsname{\let\PY@bf=\textbf\def\PY@tc##1{\textcolor[rgb]{0.00,0.50,0.00}{##1}}}
\expandafter\def\csname PY@tok@s\endcsname{\def\PY@tc##1{\textcolor[rgb]{0.73,0.13,0.13}{##1}}}
\expandafter\def\csname PY@tok@kp\endcsname{\def\PY@tc##1{\textcolor[rgb]{0.00,0.50,0.00}{##1}}}
\expandafter\def\csname PY@tok@w\endcsname{\def\PY@tc##1{\textcolor[rgb]{0.73,0.73,0.73}{##1}}}
\expandafter\def\csname PY@tok@kt\endcsname{\def\PY@tc##1{\textcolor[rgb]{0.69,0.00,0.25}{##1}}}
\expandafter\def\csname PY@tok@ow\endcsname{\let\PY@bf=\textbf\def\PY@tc##1{\textcolor[rgb]{0.67,0.13,1.00}{##1}}}
\expandafter\def\csname PY@tok@sb\endcsname{\def\PY@tc##1{\textcolor[rgb]{0.73,0.13,0.13}{##1}}}
\expandafter\def\csname PY@tok@k\endcsname{\let\PY@bf=\textbf\def\PY@tc##1{\textcolor[rgb]{0.00,0.50,0.00}{##1}}}
\expandafter\def\csname PY@tok@se\endcsname{\let\PY@bf=\textbf\def\PY@tc##1{\textcolor[rgb]{0.73,0.40,0.13}{##1}}}
\expandafter\def\csname PY@tok@sd\endcsname{\let\PY@it=\textit\def\PY@tc##1{\textcolor[rgb]{0.73,0.13,0.13}{##1}}}

\def\PYZbs{\char`\\}
\def\PYZus{\char`\_}
\def\PYZob{\char`\{}
\def\PYZcb{\char`\}}
\def\PYZca{\char`\^}
\def\PYZam{\char`\&}
\def\PYZlt{\char`\<}
\def\PYZgt{\char`\>}
\def\PYZsh{\char`\#}
\def\PYZpc{\char`\%}
\def\PYZdl{\char`\$}
\def\PYZhy{\char`\-}
\def\PYZsq{\char`\'}
\def\PYZdq{\char`\"}
\def\PYZti{\char`\~}
% for compatibility with earlier versions
\def\PYZat{@}
\def\PYZlb{[}
\def\PYZrb{]}
\makeatother


    % Exact colors from NB
    \definecolor{incolor}{rgb}{0.0, 0.0, 0.5}
    \definecolor{outcolor}{rgb}{0.545, 0.0, 0.0}



    
    % Prevent overflowing lines due to hard-to-break entities
    \sloppy 
    % Setup hyperref package
    \hypersetup{
      breaklinks=true,  % so long urls are correctly broken across lines
      colorlinks=true,
      urlcolor=blue,
      linkcolor=darkorange,
      citecolor=darkgreen,
      }
    % Slightly bigger margins than the latex defaults
    
    \geometry{verbose,tmargin=1in,bmargin=1in,lmargin=1in,rmargin=1in}
    
    

    \begin{document}
    
    
    \maketitle
    
    

    

    \section{CSE891 Homework 1}


    Camille Welcher

    \begin{Verbatim}[commandchars=\\\{\}]
{\color{incolor}In [{\color{incolor}3}]:} \PY{k+kn}{import} \PY{n+nn}{pandas} \PY{k+kn}{as} \PY{n+nn}{pd}
        \PY{k+kn}{import} \PY{n+nn}{numpy} \PY{k+kn}{as} \PY{n+nn}{np}
        \PY{k+kn}{import} \PY{n+nn}{seaborn} \PY{k+kn}{as} \PY{n+nn}{sns}
\end{Verbatim}


    \subsection{Problem 1}


    \emph{Consider a multi-core CPU with $p$ cores. The peak
single-precision performance of each core is 10 GFlops/s. The peak
bandwidth for the memory bus is 60 GB/s. Assume that you are evaluating
a polynomial of the form, $x = y^2+z^3+yz$, which can be implemented as
follows:}

\begin{verbatim}
    float x[N], y[N], z[N];
    for (i=0; i < N; ++i)
        x[i] = y[i]*y[i] + z[i]*z[i]*z[i] + y[i]*z[i];
\end{verbatim}


    \subsubsection{Part A}


    \emph{What is the arithmetic intensity (flops/byte) of each iteration of
the loop? Think of how many bytes need to be transferred over the memory
bus (both reads and writes) and how many floating point operations are
performed.}

For each iteration, there are 6 floating point ops (4 mults, 2 adds).
There is 1 write (the assignment of \texttt{x{[}i{]}}) and there are 2
reads (\texttt{y{[}i{]}} and \texttt{z{[}i{]}}), assuming that once
we've fetched a value from memory, we can use the cached value for
repeated uses; with single precision, each r/w is 4 bytes, so we get 12
bytes total. This means that our arithmetic intensity is
$6/12 = .5 \text{ flops/byte}$.


    \subsubsection{Part B}


    \emph{Plot the theoretical peak performance of the system (GFlops/s) as
a function of the number of cores, where $1 \leq p \leq 8$. For what
values of p would this computation's performance be compute-bound and
for what values would it be memory-bound?}

    \begin{Verbatim}[commandchars=\\\{\}]
{\color{incolor}In [{\color{incolor}4}]:} \PY{n}{X\PYZus{}p} \PY{o}{=} \PY{n}{np}\PY{o}{.}\PY{n}{arange}\PY{p}{(}\PY{l+m+mi}{1}\PY{p}{,}\PY{l+m+mi}{9}\PY{p}{)}
        \PY{n}{flops\PYZus{}func} \PY{o}{=} \PY{n}{np}\PY{o}{.}\PY{n}{vectorize}\PY{p}{(}\PY{k}{lambda} \PY{n}{x}\PY{p}{:} \PY{n+nb}{min}\PY{p}{(}\PY{n}{x} \PY{o}{*} \PY{l+m+mi}{10}\PY{p}{,} \PY{l+m+mi}{60} \PY{o}{*} \PY{o}{.}\PY{l+m+mi}{5}\PY{p}{)}\PY{p}{)}
        \PY{n}{Y\PYZus{}f} \PY{o}{=} \PY{n}{flops\PYZus{}func}\PY{p}{(}\PY{n}{X\PYZus{}p}\PY{p}{)}
        \PY{n}{plot}\PY{p}{(}\PY{n}{X\PYZus{}p}\PY{p}{,} \PY{n}{Y\PYZus{}f}\PY{p}{)}
        \PY{n}{xlabel}\PY{p}{(}\PY{l+s}{\PYZsq{}}\PY{l+s}{Cores}\PY{l+s}{\PYZsq{}}\PY{p}{)}
        \PY{n}{ylabel}\PY{p}{(}\PY{l+s}{\PYZsq{}}\PY{l+s}{Peak GFlops/S}\PY{l+s}{\PYZsq{}}\PY{p}{)}
        \PY{n}{title}\PY{p}{(}\PY{l+s}{\PYZsq{}}\PY{l+s}{Max GFlop/S as a function of Cores}\PY{l+s}{\PYZsq{}}\PY{p}{)}
        \PY{n}{axis}\PY{p}{(}\PY{n}{ymax}\PY{o}{=}\PY{l+m+mi}{50}\PY{p}{,} \PY{n}{ymin}\PY{o}{=}\PY{l+m+mi}{0}\PY{p}{)}
        \PY{n}{savefig}\PY{p}{(}\PY{l+s}{\PYZsq{}}\PY{l+s}{1a.png}\PY{l+s}{\PYZsq{}}\PY{p}{)}
\end{Verbatim}

    \begin{center}
    \adjustimage{max size={0.9\linewidth}{0.9\paperheight}}{Homework_1_files/Homework_1_9_0.png}
    \end{center}
    { \hspace*{\fill} \\}
    
    In a completely ideal system, GFlops/S would continue to scale with the
number of cores until saturating the memory bus; there are 2 bytes of
memory used per 1 flop, so we saturate the memory bus at 3 cores and 30
GFlops/S. Under 3 cores, we are compute bound, and over, we are memory
bound.


    \subsubsection{Part C}


    \emph{Repeat A and B for the same computation but this time in
double-precision (i.e.~double x{[}N{]}, y{[}N{]}, z{[}N{]}). The peak
double-precision performance per core is half of its single- precision
peak - 5 GFlops/s.}

Now, $q=0.25$, as we've doubled the number of bytes. We still saturate
the memory bus at 3 cores, but peak performance is now limited to 15
GFlops/S.

    \begin{Verbatim}[commandchars=\\\{\}]
{\color{incolor}In [{\color{incolor}5}]:} \PY{n}{X\PYZus{}p} \PY{o}{=} \PY{n}{np}\PY{o}{.}\PY{n}{arange}\PY{p}{(}\PY{l+m+mi}{1}\PY{p}{,}\PY{l+m+mi}{9}\PY{p}{)}
        \PY{n}{flops\PYZus{}func} \PY{o}{=} \PY{n}{np}\PY{o}{.}\PY{n}{vectorize}\PY{p}{(}\PY{k}{lambda} \PY{n}{x}\PY{p}{:} \PY{n+nb}{min}\PY{p}{(}\PY{n}{x} \PY{o}{*} \PY{l+m+mi}{5}\PY{p}{,} \PY{l+m+mi}{60} \PY{o}{*} \PY{o}{.}\PY{l+m+mi}{25}\PY{p}{)}\PY{p}{)}
        \PY{n}{Y\PYZus{}f} \PY{o}{=} \PY{n}{flops\PYZus{}func}\PY{p}{(}\PY{n}{X\PYZus{}p}\PY{p}{)}
        \PY{n}{plot}\PY{p}{(}\PY{n}{X\PYZus{}p}\PY{p}{,} \PY{n}{Y\PYZus{}f}\PY{p}{)}
        \PY{n}{xlabel}\PY{p}{(}\PY{l+s}{\PYZsq{}}\PY{l+s}{Cores}\PY{l+s}{\PYZsq{}}\PY{p}{)}
        \PY{n}{ylabel}\PY{p}{(}\PY{l+s}{\PYZsq{}}\PY{l+s}{Peak GFlops/S}\PY{l+s}{\PYZsq{}}\PY{p}{)}
        \PY{n}{title}\PY{p}{(}\PY{l+s}{\PYZsq{}}\PY{l+s}{Max GFlop/S as a function of Cores}\PY{l+s}{\PYZsq{}}\PY{p}{)}
        \PY{n}{axis}\PY{p}{(}\PY{n}{ymax}\PY{o}{=}\PY{l+m+mi}{50}\PY{p}{,} \PY{n}{ymin}\PY{o}{=}\PY{l+m+mi}{0}\PY{p}{)}
        \PY{n}{savefig}\PY{p}{(}\PY{l+s}{\PYZsq{}}\PY{l+s}{1c.png}\PY{l+s}{\PYZsq{}}\PY{p}{)}
\end{Verbatim}

    \begin{center}
    \adjustimage{max size={0.9\linewidth}{0.9\paperheight}}{Homework_1_files/Homework_1_13_0.png}
    \end{center}
    { \hspace*{\fill} \\}
    

    \subsection{Problem 2}


    \emph{Consider the same architecture as above - $p$ cores, each with 5
GFlops/s peak double- precision performance and 60 GB/s memory
bandwidth. Let there be only a single level of cache with a total
bandwidth of 300 GB/s (from cache to all cores). Now assume that your
computational kernel has an arithmetic intensity of 0.125 flops/byte and
a cache hit ratio of 50\%. Plot the performance of the system as a
function of the number of cores p.~Denote the ranges for p, where the
peak performance is bound by the CPU, the memory bandwidth and the cache
bandwidth (if applicable).}

    \begin{Verbatim}[commandchars=\\\{\}]
{\color{incolor}In [{\color{incolor}6}]:} \PY{n}{B\PYZus{}cache} \PY{o}{=} \PY{l+m+mf}{300.0}
        \PY{n}{B\PYZus{}mem} \PY{o}{=} \PY{l+m+mf}{60.0}
        \PY{n}{F\PYZus{}p} \PY{o}{=} \PY{l+m+mf}{5.0}
        \PY{n}{q} \PY{o}{=} \PY{o}{.}\PY{l+m+mi}{125}
        
        \PY{n}{X\PYZus{}p} \PY{o}{=} \PY{n}{np}\PY{o}{.}\PY{n}{arange}\PY{p}{(}\PY{l+m+mi}{1}\PY{p}{,}\PY{l+m+mi}{9}\PY{p}{)}
        
        \PY{k}{def} \PY{n+nf}{flops\PYZus{}func}\PY{p}{(}\PY{n}{ncores}\PY{p}{,} \PY{n}{cmr}\PY{o}{=}\PY{o}{.}\PY{l+m+mi}{5}\PY{p}{)}\PY{p}{:}
        
            \PY{n}{cpu\PYZus{}perf} \PY{o}{=} \PY{n}{ncores} \PY{o}{*} \PY{n}{F\PYZus{}p}
            \PY{n+nb}{bytes} \PY{o}{=} \PY{n+nb}{float}\PY{p}{(}\PY{n}{cpu\PYZus{}perf}\PY{p}{)} \PY{o}{/} \PY{n}{q}
            \PY{n}{max\PYZus{}cache\PYZus{}bytes} \PY{o}{=} \PY{n}{B\PYZus{}cache} \PY{o}{*} \PY{n}{cmr}
            \PY{k}{if} \PY{n+nb}{bytes} \PY{o}{\PYZlt{}} \PY{n}{max\PYZus{}cache\PYZus{}bytes}\PY{p}{:}
                \PY{k}{return} \PY{n}{cpu\PYZus{}perf}
            \PY{n}{bytes\PYZus{}to\PYZus{}mem} \PY{o}{=} \PY{n+nb}{bytes} \PY{o}{\PYZhy{}} \PY{n}{max\PYZus{}cache\PYZus{}bytes}
            \PY{k}{if} \PY{n}{bytes\PYZus{}to\PYZus{}mem} \PY{o}{\PYZlt{}} \PY{n}{B\PYZus{}mem}\PY{p}{:}
                \PY{k}{return} \PY{n}{max\PYZus{}cache\PYZus{}bytes} \PY{o}{*} \PY{n}{q} \PY{o}{+} \PY{n}{bytes\PYZus{}to\PYZus{}mem} \PY{o}{*} \PY{n}{q}
            \PY{k}{else}\PY{p}{:}
                \PY{k}{return} \PY{n}{max\PYZus{}cache\PYZus{}bytes} \PY{o}{*} \PY{n}{q} \PY{o}{+} \PY{n}{B\PYZus{}mem} \PY{o}{*} \PY{n}{q}
        
        \PY{n}{v\PYZus{}flops\PYZus{}func} \PY{o}{=} \PY{n}{np}\PY{o}{.}\PY{n}{vectorize}\PY{p}{(}\PY{n}{flops\PYZus{}func}\PY{p}{)}
        
        \PY{k}{for} \PY{n}{cmr} \PY{o+ow}{in} \PY{p}{[}\PY{o}{.}\PY{l+m+mi}{5}\PY{p}{,} \PY{o}{.}\PY{l+m+mi}{7}\PY{p}{,} \PY{o}{.}\PY{l+m+mi}{9}\PY{p}{]}\PY{p}{:}
            \PY{n}{Y\PYZus{}f} \PY{o}{=} \PY{n}{v\PYZus{}flops\PYZus{}func}\PY{p}{(}\PY{n}{X\PYZus{}p}\PY{p}{,} \PY{n}{cmr}\PY{o}{=}\PY{n}{cmr}\PY{p}{)}
            \PY{n}{plot}\PY{p}{(}\PY{n}{X\PYZus{}p}\PY{p}{,} \PY{n}{Y\PYZus{}f}\PY{p}{,} \PY{n}{label}\PY{o}{=}\PY{l+s}{\PYZsq{}}\PY{l+s}{Cache hit ratio=\PYZob{}\PYZcb{}}\PY{l+s}{\PYZsq{}}\PY{o}{.}\PY{n}{format}\PY{p}{(}\PY{n}{cmr}\PY{p}{)}\PY{p}{)}
        
        \PY{n}{vlines}\PY{p}{(}\PY{l+m+mi}{5}\PY{p}{,} \PY{l+m+mi}{0}\PY{p}{,} \PY{l+m+mi}{50}\PY{p}{,} \PY{n}{label}\PY{o}{=}\PY{l+s}{\PYZsq{}}\PY{l+s}{cache bound}\PY{l+s}{\PYZsq{}}\PY{p}{)}
            
        \PY{n}{xlabel}\PY{p}{(}\PY{l+s}{\PYZsq{}}\PY{l+s}{Cores}\PY{l+s}{\PYZsq{}}\PY{p}{)}
        \PY{n}{ylabel}\PY{p}{(}\PY{l+s}{\PYZsq{}}\PY{l+s}{Peak GFlops/S}\PY{l+s}{\PYZsq{}}\PY{p}{)}
        \PY{n}{title}\PY{p}{(}\PY{l+s}{\PYZsq{}}\PY{l+s}{Max GFlop/S as a function of Cores}\PY{l+s}{\PYZsq{}}\PY{p}{)}
        \PY{n}{axis}\PY{p}{(}\PY{n}{ymax}\PY{o}{=}\PY{l+m+mi}{50}\PY{p}{,} \PY{n}{ymin}\PY{o}{=}\PY{l+m+mi}{0}\PY{p}{)}
        \PY{n}{legend}\PY{p}{(}\PY{p}{)}
        \PY{n}{savefig}\PY{p}{(}\PY{l+s}{\PYZsq{}}\PY{l+s}{2.png}\PY{l+s}{\PYZsq{}}\PY{p}{)}
\end{Verbatim}

    \begin{center}
    \adjustimage{max size={0.9\linewidth}{0.9\paperheight}}{Homework_1_files/Homework_1_16_0.png}
    \end{center}
    { \hspace*{\fill} \\}
    

    \subsection{Problem 3}


    \emph{Assume you are the director of Simulation University's HPC center
and you are given a \$1 million budget (you may go over the budget by up
to 2\% if you have a good justification to do so) to purchase a new
system. You did some market research and found out that the compute
nodes you like cost \$5,000/node and each interconnect link (together
with the necessary network hardware) costs \$1000/piece. You did a
survey among the HPC center! users and (on average) they rated the
utility of an additional compute node as 5 and the utility of increasing
the bisection bandwidth of the network by one link as 3. Think of the
network topologies we discussed in class - bus, ring, 2D/3D torus (or
k-D torus), hypercube,tree or a fat-tree. How many nodes would you buy
and how would you connect them? What if the users rate the compute
vs.~communicate utilities as 5 vs 1?}


    \subsubsection{Utility = (5,3)}


    \begin{Verbatim}[commandchars=\\\{\}]
{\color{incolor}In [{\color{incolor}11}]:} \PY{n}{B} \PY{o}{=} \PY{l+m+mf}{1000000.0}
\end{Verbatim}

    \begin{Verbatim}[commandchars=\\\{\}]
{\color{incolor}In [{\color{incolor}12}]:} \PY{k}{def} \PY{n+nf}{utility}\PY{p}{(}\PY{n}{D}\PY{p}{,} \PY{n}{b}\PY{p}{,} \PY{n}{u\PYZus{}d}\PY{o}{=}\PY{l+m+mi}{5}\PY{p}{,} \PY{n}{u\PYZus{}b}\PY{o}{=}\PY{l+m+mi}{3}\PY{p}{)}\PY{p}{:}
             \PY{k}{return} \PY{n}{D} \PY{o}{*} \PY{n}{u\PYZus{}d} \PY{o}{+} \PY{n}{b} \PY{o}{*} \PY{n}{u\PYZus{}b}
\end{Verbatim}

    \begin{Verbatim}[commandchars=\\\{\}]
{\color{incolor}In [{\color{incolor}13}]:} \PY{k}{def} \PY{n+nf}{cost\PYZus{}ring}\PY{p}{(}\PY{n}{p}\PY{p}{,} \PY{n}{cp}\PY{o}{=}\PY{l+m+mi}{5000}\PY{p}{,} \PY{n}{cl}\PY{o}{=}\PY{l+m+mi}{1000}\PY{p}{)}\PY{p}{:}
             \PY{k}{return} \PY{n+nb}{int}\PY{p}{(}\PY{n}{p}\PY{p}{)} \PY{o}{*} \PY{n}{cp} \PY{o}{+} \PY{n}{cl} \PY{o}{*} \PY{n}{p}
         \PY{c}{\PYZsh{} return D, edges, bisect}
         \PY{k}{def} \PY{n+nf}{stat\PYZus{}ring}\PY{p}{(}\PY{n}{p}\PY{p}{)}\PY{p}{:}
             \PY{k}{return} \PY{n+nb}{int}\PY{p}{(}\PY{n}{p}\PY{p}{)} \PY{o}{/} \PY{l+m+mi}{2}\PY{p}{,} \PY{n}{p}\PY{p}{,} \PY{l+m+mi}{2}
         \PY{n}{v\PYZus{}cost\PYZus{}ring} \PY{o}{=} \PY{n}{np}\PY{o}{.}\PY{n}{vectorize}\PY{p}{(}\PY{n}{cost\PYZus{}ring}\PY{p}{)}
\end{Verbatim}

    \begin{Verbatim}[commandchars=\\\{\}]
{\color{incolor}In [{\color{incolor}14}]:} \PY{n}{X\PYZus{}p} \PY{o}{=} \PY{n}{np}\PY{o}{.}\PY{n}{arange}\PY{p}{(}\PY{l+m+mi}{0}\PY{p}{,}\PY{l+m+mi}{300}\PY{p}{)}
         \PY{n}{c} \PY{o}{=} \PY{n}{v\PYZus{}cost\PYZus{}ring}\PY{p}{(}\PY{n}{X\PYZus{}p}\PY{p}{)}
         \PY{n}{plot}\PY{p}{(}\PY{n}{X\PYZus{}p}\PY{p}{,} \PY{n}{c}\PY{p}{)}
         \PY{n}{axis}\PY{p}{(}\PY{n}{ymax}\PY{o}{=}\PY{n}{B}\PY{p}{)}
         \PY{n}{max\PYZus{}p\PYZus{}ring} \PY{o}{=} \PY{n}{np}\PY{o}{.}\PY{n}{argmax}\PY{p}{(}\PY{n}{c}\PY{p}{[}\PY{n}{c} \PY{o}{\PYZlt{}}\PY{o}{=} \PY{n}{B}\PY{p}{]}\PY{p}{)}
         \PY{k}{print} \PY{l+s}{\PYZsq{}}\PY{l+s}{Max nodes for budget:}\PY{l+s}{\PYZsq{}}\PY{p}{,} \PY{n}{max\PYZus{}p\PYZus{}ring}
         \PY{n}{savefig}\PY{p}{(}\PY{l+s}{\PYZsq{}}\PY{l+s}{3a.png}\PY{l+s}{\PYZsq{}}\PY{p}{)}
\end{Verbatim}

    \begin{Verbatim}[commandchars=\\\{\}]
Max nodes for budget: 166
    \end{Verbatim}

    \begin{center}
    \adjustimage{max size={0.9\linewidth}{0.9\paperheight}}{Homework_1_files/Homework_1_23_1.png}
    \end{center}
    { \hspace*{\fill} \\}
    
    \begin{Verbatim}[commandchars=\\\{\}]
{\color{incolor}In [{\color{incolor}15}]:} \PY{n}{D}\PY{p}{,} \PY{n}{e}\PY{p}{,} \PY{n}{b} \PY{o}{=} \PY{n}{stat\PYZus{}ring}\PY{p}{(}\PY{n}{max\PYZus{}p\PYZus{}ring}\PY{p}{)}
         \PY{k}{print} \PY{l+s}{\PYZsq{}}\PY{l+s}{Utility of ring with \PYZob{}\PYZcb{} nodes:}\PY{l+s}{\PYZsq{}}\PY{o}{.}\PY{n}{format}\PY{p}{(}\PY{n}{max\PYZus{}p\PYZus{}ring}\PY{p}{)}\PY{p}{,} \PY{n}{utility}\PY{p}{(}\PY{n}{D}\PY{p}{,} \PY{n}{b}\PY{p}{)}
\end{Verbatim}

    \begin{Verbatim}[commandchars=\\\{\}]
Utility of ring with 166 nodes: 421
    \end{Verbatim}

    \begin{Verbatim}[commandchars=\\\{\}]
{\color{incolor}In [{\color{incolor}16}]:} \PY{k}{def} \PY{n+nf}{cost\PYZus{}torus}\PY{p}{(}\PY{n}{p}\PY{p}{,} \PY{n}{k}\PY{p}{,} \PY{n}{cp}\PY{o}{=}\PY{l+m+mi}{5000}\PY{p}{,} \PY{n}{cl}\PY{o}{=}\PY{l+m+mi}{1000}\PY{p}{)}\PY{p}{:}
             \PY{k}{return} \PY{n+nb}{int}\PY{p}{(}\PY{n}{p}\PY{p}{)} \PY{o}{*} \PY{n}{cp} \PY{o}{+} \PY{n}{k} \PY{o}{*} \PY{n}{cl} \PY{o}{*} \PY{n}{p}
         \PY{c}{\PYZsh{} return D, edges, bisect}
         \PY{k}{def} \PY{n+nf}{stat\PYZus{}torus}\PY{p}{(}\PY{n}{p}\PY{p}{,} \PY{n}{k}\PY{p}{)}\PY{p}{:}
             \PY{k}{return} \PY{n+nb}{int}\PY{p}{(}\PY{n}{k}\PY{p}{)} \PY{o}{*} \PY{n+nb}{int}\PY{p}{(}\PY{n}{p}\PY{p}{)}\PY{o}{/}\PY{l+m+mi}{2}\PY{p}{,} \PY{n+nb}{int}\PY{p}{(}\PY{n}{k}\PY{p}{)} \PY{o}{*} \PY{n+nb}{int}\PY{p}{(}\PY{n}{p}\PY{p}{)}\PY{p}{,} \PY{l+m+mi}{2} \PY{o}{*} \PY{n}{p}
         \PY{n}{v\PYZus{}cost\PYZus{}torus} \PY{o}{=} \PY{n}{np}\PY{o}{.}\PY{n}{vectorize}\PY{p}{(}\PY{n}{cost\PYZus{}torus}\PY{p}{)}
\end{Verbatim}

    \begin{Verbatim}[commandchars=\\\{\}]
{\color{incolor}In [{\color{incolor}17}]:} \PY{n}{X\PYZus{}p} \PY{o}{=} \PY{n}{np}\PY{o}{.}\PY{n}{arange}\PY{p}{(}\PY{l+m+mi}{0}\PY{p}{,}\PY{l+m+mi}{300}\PY{p}{)}
         \PY{n}{max\PYZus{}p\PYZus{}torus} \PY{o}{=} \PY{p}{[}\PY{p}{]}
         \PY{k}{for} \PY{n}{k} \PY{o+ow}{in} \PY{n+nb}{range}\PY{p}{(}\PY{l+m+mi}{2}\PY{p}{,}\PY{l+m+mi}{6}\PY{p}{)}\PY{p}{:}
             \PY{n}{c} \PY{o}{=} \PY{n}{v\PYZus{}cost\PYZus{}torus}\PY{p}{(}\PY{n}{X\PYZus{}p}\PY{p}{,} \PY{n}{k}\PY{p}{)}
             \PY{n}{plot}\PY{p}{(}\PY{n}{X\PYZus{}p}\PY{p}{,} \PY{n}{c}\PY{p}{,} \PY{n}{label}\PY{o}{=}\PY{l+s}{\PYZsq{}}\PY{l+s}{k=\PYZob{}\PYZcb{}}\PY{l+s}{\PYZsq{}}\PY{o}{.}\PY{n}{format}\PY{p}{(}\PY{n}{k}\PY{p}{)}\PY{p}{)}
             \PY{n}{m} \PY{o}{=} \PY{n}{np}\PY{o}{.}\PY{n}{argmax}\PY{p}{(}\PY{n}{c}\PY{p}{[}\PY{n}{c} \PY{o}{\PYZlt{}}\PY{o}{=} \PY{n}{B}\PY{p}{]}\PY{p}{)}
             \PY{n}{max\PYZus{}p\PYZus{}torus}\PY{o}{.}\PY{n}{append}\PY{p}{(}\PY{p}{(}\PY{n}{m}\PY{p}{,} \PY{n}{k}\PY{p}{)}\PY{p}{)}
             \PY{k}{print} \PY{l+s}{\PYZsq{}}\PY{l+s}{Max nodes for budget at k=\PYZob{}\PYZcb{}:}\PY{l+s}{\PYZsq{}}\PY{o}{.}\PY{n}{format}\PY{p}{(}\PY{n}{k}\PY{p}{)}\PY{p}{,} \PY{n}{m}
         \PY{n}{axis}\PY{p}{(}\PY{n}{ymax}\PY{o}{=}\PY{n}{B}\PY{p}{)}
         \PY{n}{legend}\PY{p}{(}\PY{p}{)}
         \PY{n}{savefig}\PY{p}{(}\PY{l+s}{\PYZsq{}}\PY{l+s}{3b.png}\PY{l+s}{\PYZsq{}}\PY{p}{)}
\end{Verbatim}

    \begin{Verbatim}[commandchars=\\\{\}]
Max nodes for budget at k=2: 142
Max nodes for budget at k=3: 125
Max nodes for budget at k=4: 111
Max nodes for budget at k=5: 100
    \end{Verbatim}

    \begin{center}
    \adjustimage{max size={0.9\linewidth}{0.9\paperheight}}{Homework_1_files/Homework_1_26_1.png}
    \end{center}
    { \hspace*{\fill} \\}
    
    \begin{Verbatim}[commandchars=\\\{\}]
{\color{incolor}In [{\color{incolor}18}]:} \PY{k}{for} \PY{n}{p}\PY{p}{,} \PY{n}{k} \PY{o+ow}{in} \PY{n}{max\PYZus{}p\PYZus{}torus}\PY{p}{:}
             \PY{n}{D}\PY{p}{,} \PY{n}{e}\PY{p}{,} \PY{n}{b} \PY{o}{=} \PY{n}{stat\PYZus{}torus}\PY{p}{(}\PY{n}{p}\PY{p}{,} \PY{n}{k}\PY{p}{)}
             \PY{k}{print} \PY{l+s}{\PYZsq{}}\PY{l+s}{Utility of torus with \PYZob{}\PYZcb{} nodes, k=\PYZob{}\PYZcb{}:}\PY{l+s}{\PYZsq{}}\PY{o}{.}\PY{n}{format}\PY{p}{(}\PY{n}{p}\PY{p}{,} \PY{n}{k}\PY{p}{)}\PY{p}{,} \PY{n}{utility}\PY{p}{(}\PY{n}{D}\PY{p}{,} \PY{n}{b}\PY{p}{)}
\end{Verbatim}

    \begin{Verbatim}[commandchars=\\\{\}]
Utility of torus with 142 nodes, k=2: 1562
Utility of torus with 125 nodes, k=3: 1685
Utility of torus with 111 nodes, k=4: 1776
Utility of torus with 100 nodes, k=5: 1850
    \end{Verbatim}

    \begin{Verbatim}[commandchars=\\\{\}]
{\color{incolor}In [{\color{incolor}19}]:} \PY{k}{def} \PY{n+nf}{cost\PYZus{}tree}\PY{p}{(}\PY{n}{p}\PY{p}{,} \PY{n}{cp}\PY{o}{=}\PY{l+m+mi}{5000}\PY{p}{,} \PY{n}{cl}\PY{o}{=}\PY{l+m+mi}{1000}\PY{p}{)}\PY{p}{:}
             \PY{k}{return} \PY{n+nb}{int}\PY{p}{(}\PY{n}{p}\PY{p}{)} \PY{o}{*} \PY{n}{cp} \PY{o}{+} \PY{n}{cl} \PY{o}{*} \PY{n}{p} \PY{o}{*} \PY{n}{log2}\PY{p}{(}\PY{n}{p}\PY{p}{)}
         \PY{c}{\PYZsh{} return D, edges, bisect}
         \PY{k}{def} \PY{n+nf}{stat\PYZus{}tree}\PY{p}{(}\PY{n}{p}\PY{p}{)}\PY{p}{:}
             \PY{k}{return} \PY{l+m+mi}{2} \PY{o}{*} \PY{n}{log2}\PY{p}{(}\PY{n}{p}\PY{p}{)}\PY{p}{,} \PY{n+nb}{int}\PY{p}{(}\PY{n}{p}\PY{p}{)} \PY{o}{*} \PY{n+nb}{int}\PY{p}{(}\PY{n}{log2}\PY{p}{(}\PY{n}{p}\PY{p}{)}\PY{p}{)}\PY{p}{,} \PY{n+nb}{int}\PY{p}{(}\PY{n}{p}\PY{o}{/}\PY{l+m+mi}{2}\PY{p}{)}
         \PY{n}{v\PYZus{}cost\PYZus{}tree} \PY{o}{=} \PY{n}{np}\PY{o}{.}\PY{n}{vectorize}\PY{p}{(}\PY{n}{cost\PYZus{}tree}\PY{p}{)}
\end{Verbatim}

    \begin{Verbatim}[commandchars=\\\{\}]
{\color{incolor}In [{\color{incolor}20}]:} \PY{n}{X\PYZus{}p} \PY{o}{=} \PY{n}{np}\PY{o}{.}\PY{n}{arange}\PY{p}{(}\PY{l+m+mi}{0}\PY{p}{,}\PY{l+m+mi}{300}\PY{p}{)}
         \PY{n}{c} \PY{o}{=} \PY{n}{v\PYZus{}cost\PYZus{}tree}\PY{p}{(}\PY{n}{X\PYZus{}p}\PY{p}{)}
         \PY{n}{plot}\PY{p}{(}\PY{n}{X\PYZus{}p}\PY{p}{,} \PY{n}{c}\PY{p}{)}
         \PY{n}{axis}\PY{p}{(}\PY{n}{ymax}\PY{o}{=}\PY{n}{B}\PY{p}{)}
         \PY{n}{max\PYZus{}p\PYZus{}tree} \PY{o}{=} \PY{n}{np}\PY{o}{.}\PY{n}{argmax}\PY{p}{(}\PY{n}{c}\PY{p}{[}\PY{n}{c} \PY{o}{\PYZlt{}}\PY{o}{=} \PY{n}{B}\PY{p}{]}\PY{p}{)}
         \PY{n}{savefig}\PY{p}{(}\PY{l+s}{\PYZsq{}}\PY{l+s}{3c.png}\PY{l+s}{\PYZsq{}}\PY{p}{)}
\end{Verbatim}

    \begin{center}
    \adjustimage{max size={0.9\linewidth}{0.9\paperheight}}{Homework_1_files/Homework_1_29_0.png}
    \end{center}
    { \hspace*{\fill} \\}
    
    \begin{Verbatim}[commandchars=\\\{\}]
{\color{incolor}In [{\color{incolor}21}]:} \PY{n}{D}\PY{p}{,} \PY{n}{e}\PY{p}{,} \PY{n}{b} \PY{o}{=} \PY{n}{stat\PYZus{}tree}\PY{p}{(}\PY{n}{max\PYZus{}p\PYZus{}tree}\PY{p}{)}
         \PY{k}{print} \PY{l+s}{\PYZsq{}}\PY{l+s}{Utility of fat tree with \PYZob{}\PYZcb{} nodes:}\PY{l+s}{\PYZsq{}}\PY{o}{.}\PY{n}{format}\PY{p}{(}\PY{n}{max\PYZus{}p\PYZus{}tree}\PY{p}{)}\PY{p}{,} \PY{n}{utility}\PY{p}{(}\PY{n}{D}\PY{p}{,} \PY{n}{b}\PY{p}{)}
\end{Verbatim}

    \begin{Verbatim}[commandchars=\\\{\}]
Utility of fat tree with 86 nodes: 193.262647547
    \end{Verbatim}

    With our budget and the given utility values, I would choose a 5-D
100-ary torus; this is 100 nodes and 500 links.


    \subsubsection{Utility = (5,1)}


    \begin{Verbatim}[commandchars=\\\{\}]
{\color{incolor}In [{\color{incolor}22}]:} \PY{n}{D}\PY{p}{,} \PY{n}{e}\PY{p}{,} \PY{n}{b} \PY{o}{=} \PY{n}{stat\PYZus{}ring}\PY{p}{(}\PY{n}{max\PYZus{}p\PYZus{}ring}\PY{p}{)}
         \PY{k}{print} \PY{l+s}{\PYZsq{}}\PY{l+s}{Utility of ring with \PYZob{}\PYZcb{} nodes:}\PY{l+s}{\PYZsq{}}\PY{o}{.}\PY{n}{format}\PY{p}{(}\PY{n}{max\PYZus{}p\PYZus{}ring}\PY{p}{)}\PY{p}{,} \PY{n}{utility}\PY{p}{(}\PY{n}{D}\PY{p}{,} \PY{n}{b}\PY{p}{,} \PY{n}{u\PYZus{}b}\PY{o}{=}\PY{l+m+mi}{1}\PY{p}{)}
\end{Verbatim}

    \begin{Verbatim}[commandchars=\\\{\}]
Utility of ring with 166 nodes: 417
    \end{Verbatim}

    \begin{Verbatim}[commandchars=\\\{\}]
{\color{incolor}In [{\color{incolor}23}]:} \PY{k}{for} \PY{n}{p}\PY{p}{,} \PY{n}{k} \PY{o+ow}{in} \PY{n}{max\PYZus{}p\PYZus{}torus}\PY{p}{:}
             \PY{n}{D}\PY{p}{,} \PY{n}{e}\PY{p}{,} \PY{n}{b} \PY{o}{=} \PY{n}{stat\PYZus{}torus}\PY{p}{(}\PY{n}{p}\PY{p}{,} \PY{n}{k}\PY{p}{)}
             \PY{k}{print} \PY{l+s}{\PYZsq{}}\PY{l+s}{Utility of torus with \PYZob{}\PYZcb{} nodes, k=\PYZob{}\PYZcb{}:}\PY{l+s}{\PYZsq{}}\PY{o}{.}\PY{n}{format}\PY{p}{(}\PY{n}{p}\PY{p}{,} \PY{n}{k}\PY{p}{)}\PY{p}{,} \PY{n}{utility}\PY{p}{(}\PY{n}{D}\PY{p}{,} \PY{n}{b}\PY{p}{,} \PY{n}{u\PYZus{}b}\PY{o}{=}\PY{l+m+mi}{1}\PY{p}{)}
\end{Verbatim}

    \begin{Verbatim}[commandchars=\\\{\}]
Utility of torus with 142 nodes, k=2: 994
Utility of torus with 125 nodes, k=3: 1185
Utility of torus with 111 nodes, k=4: 1332
Utility of torus with 100 nodes, k=5: 1450
    \end{Verbatim}

    \begin{Verbatim}[commandchars=\\\{\}]
{\color{incolor}In [{\color{incolor}24}]:} \PY{n}{D}\PY{p}{,} \PY{n}{e}\PY{p}{,} \PY{n}{b} \PY{o}{=} \PY{n}{stat\PYZus{}tree}\PY{p}{(}\PY{n}{max\PYZus{}p\PYZus{}tree}\PY{p}{)}
         \PY{k}{print} \PY{l+s}{\PYZsq{}}\PY{l+s}{Utility of fat tree with \PYZob{}\PYZcb{} nodes:}\PY{l+s}{\PYZsq{}}\PY{o}{.}\PY{n}{format}\PY{p}{(}\PY{n}{max\PYZus{}p\PYZus{}tree}\PY{p}{)}\PY{p}{,} \PY{n}{utility}\PY{p}{(}\PY{n}{D}\PY{p}{,} \PY{n}{b}\PY{p}{,} \PY{n}{u\PYZus{}b}\PY{o}{=}\PY{l+m+mi}{1}\PY{p}{)}
\end{Verbatim}

    \begin{Verbatim}[commandchars=\\\{\}]
Utility of fat tree with 86 nodes: 107.262647547
    \end{Verbatim}

    The 5-D torus still wins out, although lower dimensions inch up in
value. This analysis also doesn't factor in some of the complexity of
constructing torus networks, or the potential variable bisection
bandwidths of a fat-tree.


    \subsection{Problem 4}


    \emph{The outer product is a commonly used tensor operation. The input
is two vectors of length N and M. The result is a matrix of dimensions N
by M. In code it would look as follows:}

\begin{verbatim}
double x[N], y[M], A[N][M];
    for (i=0; i<N; ++i)
        for (j=0; j<M; ++j)
            A[i][j] = x[i]*y[j];
\end{verbatim}


    \subsubsection{Part A}


    \emph{Assume that there is only a single level of cache of size 64 KBs,
a cache line is 64 bytes and the Least Recently Used (LRU) policy is
adopted for cache eviction. For N = M = 1024, how many cache misses
would the implementation above incur?}

    \begin{Verbatim}[commandchars=\\\{\}]
{\color{incolor}In [{\color{incolor}25}]:} \PY{k}{print} \PY{l+s}{\PYZsq{}}\PY{l+s}{cache holds}\PY{l+s}{\PYZsq{}}\PY{p}{,} \PY{l+m+mi}{64000} \PY{o}{/} \PY{l+m+mi}{8}\PY{p}{,} \PY{l+s}{\PYZsq{}}\PY{l+s}{doubles, cache line is 8 doubles}\PY{l+s}{\PYZsq{}}
\end{Verbatim}

    \begin{Verbatim}[commandchars=\\\{\}]
cache holds 8000 doubles, cache line is 8 doubles
    \end{Verbatim}

    \texttt{x{[}N{]}} and \texttt{y{[}N{]}} can both reside entirely in
cache, consuming 2048 of our 8000 doubles. Misses should only occur when
writing into \texttt{A}. With a 64 byte cache line that can store 8
doubles, we get $(M * N) / 8$ misses, or 131,072.


    \subsubsection{Part B}


    \emph{Repeat A, but now for N = M = 10240.}

    Now we cannot store the entirity of \texttt{x} and \texttt{y} in cache.
Each of the $N$ iterations of the outer loop will cause a miss, and
we'll have $M / 8$ misses for accessing \texttt{y} and $M / 8$ misses
for writing into \texttt{A}, giving a total of $N * (M/4)$ misses, or
26,214,400 total.


    \subsubsection{Part C}


    \emph{Based on your insights from 4.A and 4.B, give the pseudo-code for
a high performance outer product implementation with a reduced number of
cache misses.}

    Pseudocode:

\begin{verbatim}
S = floor((cache_size / 8) / 3)
for s_i in 0..N/S:
    loadCache(x[s_i*S:(i+1)*S])
    for s_j in 0..M/S:
        loadCache(y[s_j*S:(j+1)*S])
        loadCache(A[s_i*S:(s_i+1)*S, s_j*S:(s_j+1)*S])
        for ii in s_i*S..(s_i+1)*S:
            for jj in s_j*S..(s_j+1)*S:
                A[ii,jj] = x[ii] * y[jj]
\end{verbatim}

Thus, the idea is to subdivide \texttt{A} into square submatrices, such
that we can load the ranges of \texttt{x} and \texttt{y} and single row
of the submatrix of \texttt{A} into memory for each submatrix. Traversal
order would follow the pattern of the following matrix:

    \begin{Verbatim}[commandchars=\\\{\}]
{\color{incolor}In [{\color{incolor}26}]:} \PY{n}{N} \PY{o}{=} \PY{l+m+mi}{25}
         \PY{n}{A} \PY{o}{=} \PY{n}{np}\PY{o}{.}\PY{n}{zeros}\PY{p}{(}\PY{p}{(}\PY{n}{N}\PY{p}{,}\PY{n}{N}\PY{p}{)}\PY{p}{)}
         \PY{n}{S} \PY{o}{=} \PY{l+m+mi}{5}
         \PY{n}{z} \PY{o}{=} \PY{l+m+mi}{0}
         \PY{k}{for} \PY{n}{s\PYZus{}i} \PY{o+ow}{in} \PY{n+nb}{range}\PY{p}{(}\PY{n}{N}\PY{o}{/}\PY{n}{S}\PY{p}{)}\PY{p}{:}
             \PY{k}{for} \PY{n}{s\PYZus{}j} \PY{o+ow}{in} \PY{n+nb}{range}\PY{p}{(}\PY{n}{N}\PY{o}{/}\PY{n}{S}\PY{p}{)}\PY{p}{:}
                 \PY{n}{z} \PY{o}{+}\PY{o}{=} \PY{l+m+mi}{1}
                 \PY{k}{for} \PY{n}{ii} \PY{o+ow}{in} \PY{n+nb}{range}\PY{p}{(}\PY{n}{s\PYZus{}i} \PY{o}{*} \PY{n}{S}\PY{p}{,} \PY{p}{(}\PY{n}{s\PYZus{}i} \PY{o}{+} \PY{l+m+mi}{1}\PY{p}{)} \PY{o}{*} \PY{n}{S}\PY{p}{)}\PY{p}{:}
                     \PY{k}{for} \PY{n}{jj} \PY{o+ow}{in} \PY{n+nb}{range}\PY{p}{(}\PY{n}{s\PYZus{}j} \PY{o}{*} \PY{n}{S}\PY{p}{,} \PY{p}{(}\PY{n}{s\PYZus{}j} \PY{o}{+} \PY{l+m+mi}{1}\PY{p}{)} \PY{o}{*} \PY{n}{S}\PY{p}{)}\PY{p}{:}
                         \PY{n}{A}\PY{p}{[}\PY{n}{ii}\PY{p}{,}\PY{n}{jj}\PY{p}{]} \PY{o}{=} \PY{n}{z}
\end{Verbatim}

    \begin{Verbatim}[commandchars=\\\{\}]
{\color{incolor}In [{\color{incolor}27}]:} \PY{n}{A}\PY{p}{[}\PY{l+m+mi}{0}\PY{p}{:}\PY{l+m+mi}{10}\PY{p}{,}\PY{l+m+mi}{0}\PY{p}{:}\PY{l+m+mi}{10}\PY{p}{]}
\end{Verbatim}

            \begin{Verbatim}[commandchars=\\\{\}]
{\color{outcolor}Out[{\color{outcolor}27}]:} array([[ 1.,  1.,  1.,  1.,  1.,  2.,  2.,  2.,  2.,  2.],
                [ 1.,  1.,  1.,  1.,  1.,  2.,  2.,  2.,  2.,  2.],
                [ 1.,  1.,  1.,  1.,  1.,  2.,  2.,  2.,  2.,  2.],
                [ 1.,  1.,  1.,  1.,  1.,  2.,  2.,  2.,  2.,  2.],
                [ 1.,  1.,  1.,  1.,  1.,  2.,  2.,  2.,  2.,  2.],
                [ 6.,  6.,  6.,  6.,  6.,  7.,  7.,  7.,  7.,  7.],
                [ 6.,  6.,  6.,  6.,  6.,  7.,  7.,  7.,  7.,  7.],
                [ 6.,  6.,  6.,  6.,  6.,  7.,  7.,  7.,  7.,  7.],
                [ 6.,  6.,  6.,  6.,  6.,  7.,  7.,  7.,  7.,  7.],
                [ 6.,  6.,  6.,  6.,  6.,  7.,  7.,  7.,  7.,  7.]])
\end{Verbatim}
        

    % Add a bibliography block to the postdoc
    
    
    
    \end{document}
